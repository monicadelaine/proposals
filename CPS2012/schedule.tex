
%\section{Schedule}\label{sec:schedule}
\section{Project methodology, management, and dissemination plan}
\subsection{Team Member Roles, Responsibilities and Expertise}
%describe the roles, responsibilities, and expertise of the team members, how they cover the set of skills needed to realize the project goals and how their interactions will contribute to integration across core CPS disciplinary areas;

The success of the proposed research requires both breadth and depth in robotics hardware and software engineering.  The PI's main research area centers on distributed robot systems.  However, her twelve years as a software engineer (culminating in a IT Architect position at IBM) has made the shortcomings of existing tools apparent.  She designed the circuit board and the software for the eROSI, a novel robot platform \cite{walter2007design}.  This platform featured an ASCII message-based API available via Bluetooth and images were served wirelessly via UHF channels.  These choices directly affected the platform usability for research students such as the undergraduate team at Berea College \cite{Isaacs2006}.  In addition, the PI has either programmed or modified drivers or firmware for robots and devices including the Koala, Finch, Create, Ar.Drone, Lego Mindstorms, Calliope (Dynamixel-based 4 DOF arm), Handyboard, GPS modules, lasers, analog and digital IR and ultrasonic sensors, and custom boards using ATMega chip and serial Bluetooth ICs to interface with Player/Stage, Alice/PREOP, ROS, Carmen, Webots and custom firmware applications.  In addition, the PI has studied open source implementations of other drivers and frameworks including PSOS (Pioneer robots), Tekkotsu, and LCM.  She was the organizer of the 2010 AAAI Robotics Workshop titled ``Enabling Intelligence through Middleware'' an effort to increase awareness of tools within the community and tool needs outside the community \cite{MonicaAnderson2011}.

The engineering of modeling language and transformation is contingent to the success of the project.
Co-PI Syriani's expertise in model-based design is thus critical to the project.
He is the author of the T-Core~\cite{Syriani2010} and MoTif~\cite{Syriani2011} transformation languages.
In addition, the Co-PI has ample experience with existing modeling and transformation languages based on the Eclipse Modeling Framework (EMF).
The PIs are therefore in a unique position given their expertise, experience and enthusiasm to affect accessibility of robotics controller development within the larger community.


\subsection{Dissemination Plan}
Broad dissemination plans include both the academic and non-academic communities.  Traditional academic dissemination plans include publications and conference presentations in both the educational and robotics software engineering communities.  Future plans include workshops at conference venues as well as at partner campuses will provide hands-on opportunities for those that wish to use the tool for teaching or design.  A website will be maintained and all software will be open source.  However, primary means of dissemination will utilize partnerships with manufacturers or tool developers as we have seen this method is effective at making hardware widely available.  The PIs will also hold hands-on workshops both to inform and to gather feedback from the community.  

Additional kinematic designs are motivated by partnerships with NASA and RTP.  NASA is migrating a robotic platform from VxWorks to Ubuntu with a future update to include RTLinux.  This custom kinematic chain involves manipulators with haptic sensors on a MLVDS bus (a new transport).  RTP (Huntsville Research Technology Park) is an initiative by the governor of Alabama to create a National Robotics Technology Development and Training Center at Calhoun Community College.  Many robot platforms have been purchased and donated for training and are available for evaluation as candidate platforms for RDIS inclusion.  Both opportunities expand the platforms that can modeled and contribute to the domain model that can be generalized across other platforms. 

\subsection{Methodology and Timeline}
The research approach consists of iterative design and integrated usability checks.  Rather than attempt to create the research tools in a single iteration, it is better to start with a core of features and iteratively add to those features. This approach allows for the usability testing within the first 18 months as an input into the next iteration.  The schedule details the work for two graduate students for the three years: one student to focus on analysis of platforms and the evolution of RDIS and one student to focus on the graphical tool for building and incorporating the RDIS into tools and frameworks.  Undergraduate students will assist with coding, testing and support activities.  The project team will meet weekly to discuss progress on features and use cases and any design issues.  Ongoing activities include dissemination through web sites, conferences and site visits.    Target tools and languages include Eclipse, C, C++ and Python.  Target frameworks include ROS, Player/Stage, PREOP and URBI as license allows.  Target differential drive platforms include Pioneer, Koala, ER-1, custom based on android platform and RC-car base (skid-steer).  Other kinematic chains will include the Calliope and Chaira, two mobile manipulation platforms created at CMU.  Additional platforms will be evaluated based on partnerships with manufacturers and industrial labs.
Table\ref{tab:tasks} outlines the project schedule.
\monica{include the courses you teach in the table}
%
% Requires the booktabs if the memoir class is not being used
\begin{table}[htbp]
   \centering
   \begin{tabular}{|l|l|} 
   \hline
      {\bf Quarter}    & {\bf Task} \\
      \hline
      Q4 2012       &  Definition of new kinematic model (joints, links and transmissions)\\
      			&Retrofit to differential drive \\
      			&Update accompanying RDIS generators and links to frameworks;  \\
      Q1 2013       &Expand semantics expressiveness (threading, errors, etc)\\
						&Teach CS 691 on Model-based Design\\
      Q2 2013       & Evaluate current iteration of RDIS and tool against target differential drive robotics\\
     			&Identify collaborators that manufacturer sensors, actuators and platforms using differential drive\\
       Q3 2013	&Modify model, RDIS and tools to accommodate skid-steer\\
      Q4 2013       &Evaluate current iteration of RDIS and tool against target skid-steer \\
      Q1 2014       &Modify model, RDIS and tools to accommodate mobile manipulation platforms \\
						&Teach CS 691 on Model-based Design\\
      Q2 2014       &Evaluate current iteration of RDIS and tool against target mobile manipulation platforms \\
      			&Identify collaborators that discover kinematic parameters\\
      Q3 2014       & Modify model, RDIS and tools to accommodate bus transports (MLVDS) and RTLinux\\
      			&Summer partnership with NASA\\
      Q4 2014       &Evaluate current iteration of RDIS and tool against target platforms running under RTLinux\\
      Q1 2015       & Modify model, RDIS and tools to accommodate gantry robots and x-y-z robots\\
						&Teach CS 691 on Model-based Design\\
      Q2 2015       & Evaluate current iteration of RDIS and tool against target manufacturing platforms\\
      			& Identify opportunities for incorporation into manufacturing robot software platforms\\
      Q3 2015       & Publish final version RDIS, tools and adapters\\
      \hline
   \end{tabular}
   \caption{Timeline of proposed tasks.}
   \label{tab:tasks}
\end{table}
