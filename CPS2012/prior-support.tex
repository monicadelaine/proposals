
\section{Results from Prior NSF Support}\label{sec:prior-nsf}
\begin{enumerate}
\item Monica Anderson - PI for the NSF Grant: CCLI: Enhancing student motivation in the first year computing science curriculum (DUE- 0736789), 04/01/2008-08/31/2011.  This is an ongoing project that addresses retention issues due to motivation through specially designed modules that use PREOP in a CS0 or CS1 laboratory setting.  We provided PREOP workshops to teachers at the ARTSI Faculty Meeting, demonstrated PREOP at the Google education summit and added PREOP to Alice instructor training (July 2011). Three papers have been published including two conference papers and two journal papers \cite{Davis2009,wellman2009alice,Wellman2009b,Anderson}.  

\item Monica Anderson - PI for the Grant: ARTSI-Advancing Robotic Technology for Societal Impact (BPC-A- 0742123), 09/01/07-12/31/2012. This is an ongoing broadening participation grant that increases the number of students interested in computer science and robotics through both research and creative experiences. Robotics programs have been established or enhanced at 14 HBCUs. The PI has provided each school with both onsite and remote instructional support ranging from hands-on workshops on open-source robotics development environments and controller development to hardware assembly instructions. Seven summer research assistants have been hosted at UA, with one student contributing to an accepted paper \cite{Wellman2009} and resulting in numerous poster presentations.  
\item Monica Anderson - PI for the Grant: SGER: Impact of Intermittent Networks on Team-based Coverage (IIS-0846976), 09/01/2008-08/31/2009. This project investigated both surveillance in constrained environments using limited resources and methods for coordinating robot teams with limited communications capability. The novel approach of using ALUL and a QOS metric allowed for balanced target acquisition and tracking \cite{Veluchamy2010,Alexander2009,Mckenzie2010,McKenzie2009}.  

\item Monica Anderson - Co-PI for the Grant: EMT: Collaborative Research: Primate-inspired Heterogeneous Mobile and Static Sensor Networks (CNS: 0829827) 09/01/2008-08/31/2011. This project investigates bio-inspired approaches for multirobot coordination. We show that given a greedy approach to selecting the next point to search, observation information provides more relevant input to the next action selected. Experimental results show that using observation to infer state and intent to disperse nodes performs better than non-communicative and potential field based cooperation and in some cases direct communications. \cite{Wellman2009d,Dawson2011,Dawson2010,Dawson2011b,Wellman2009c}.  

\item Monica Anderson - PI for the Grant: University of Alabama Participation of EPSCOR Institution in SSR-RC: QOS service metrics for unmanned surveillance (IIP: 1026528).  06/15/2010-06/14/2012. We consider ALUL as a metric that indicates the target acquisition ability of a particular system configuration. When used in conjunction with temporal constraints, we can intelligently automate camera coverage to improve target acquisition and tracking performance of the system. The results from the experiments confirm that the coverage in constrained environments when the existing camera configuration cannot view large portions of the region of our interest improves when ALUL is considered.  To date, one conference paper and one journal article have been accepted \cite{Veluchamy2011,Dukeman2011}.
\item Eugene Syriani - New investigator and has no prior NSF support.

\end{enumerate}
