
\section{Research Plan}
\label{sec:research-plan}
%include a plan for validation of the research by experimentation and prototyping;

Implementation Pathways: Embedding in Hardware: Although there are many benefits to utilizing the RDIS outside of the hardware� firmware, the ultimate goal to provide accessible programming is best met through embedding the robot device descriptions within the device.  Discovery occurs when the design environment queries the device for its supported services (or APIs).  The initial approach for platforms that support onboard reconfigurable firmware is to augment the firmware to support a single command that communicates the RDIS.  The information provided by the RDIS can be used by any RDIS-enabled development environment (discussed in the next section).  It is expected that manufacturers will choose to RDS enable their devices once there are more RDIS-enabled environments are available.  The Finch designer has already agreed to update the firmware of a couple of select models and has committed to including the command once it is more stable and is being adopted (see attached letter of collaboration).  

Implementation Pathways: RDIS Creation, Validation and Driver Generation Toolset: Even as we work toward adoption by hardware manufacturers, RDIS can provide other efficiencies that also serve the purpose of increasing accessibility.   As a side effect of describing the robot device hardware communication interface, the creation of device drivers for frameworks can be automated.   This is not a trivial benefit.  Each runtime framework requires either a custom device driver that links the framework to the device or a bridge between the target framework and a framework that does have device support.   At a minimum, the RDIS relieves the manufacturer from providing point solutions for languages, development tools and frameworks.  Target toolsets include C, C++, Java, Python, ROS and Websocket server (provides HTML5 and Javascript support).  The biggest challenge to this task is the cross-platform support of the communication interfaces.  However, consideration for different platforms can be built into the templates using compiler directives.

It is important to note that the RDIS toolset is enabled by ANTLR and StringTemplate.  These are open source libraries that parse and process data according to grammars.  These grammars are often used to define domain specific languages that are subsequently processed either by interpreters or translators.  The sample RDIS and the translation to a C-based command line controller and a ROS node was achieved through the use of grammars and the ANTLR and StringTemplate libraries.  Although these libraries provide many built-in features, the ability to embed code to customize processing is important to using these tools effectively.

Implementation Pathways: Custom and Reconfigurable Robot Device: In the ``Sketching in Hardware'' community, many of the robotic devices are custom built by connecting actuators and sensors to netbooks or single board computers (SBCs).  Although the RDS cannot completely specify all possible interactions with every sensor and actuator, the communication to the computer and to the peripheral connection mechanisms can be queried and described.  A microprocessor-based software module, RDSConnect, would execute on the computer or SBC to act as both a development and runtime interface for the connected peripherals. In this way, RDSConnect mimics the functionality of firmware to provide a cohesive, discoverable interface to attached components.  RDSConnect interacts with graphical tools that allow building and validation of RDIS files connected using the described communication primitive in real time could be used to integrate a custom microprocessor-based mechanism with supported tools and languages.  This approach is useful for robots composed of communication-enabled parts (i.e. the Calliope with Dynamixel USB-enabled actuators) and allows custom robots built from commodity parts to be easily interfaced to development packages.

Reconfigurable hardware provides a unique challenge.  Rather than a static description, the RDIS becomes a mechanism for communicating changes.  As a longer-term goal, research methods that learn the current configuration based on embedded sensors could be appropriately applied here. Either building blocks or configurable linkages would embed proprioceptive sensors to establish kinematic parameters needed for programming.  This kinematic learning can also be accomplished by augmenting existing systems with small standalone, networked sensors that are placed at joints [24]. 

\subsection{Team Member Roles, Responsibilities and Expertise}
%describe the roles, responsibilities, and expertise of the team members, how they cover the set of skills needed to realize the project goals and how their interactions will contribute to integration across core CPS disciplinary areas;

The success of the proposed research requires both breadth and depth in robotics hardware and software.  The PIs main research area centers on distributed robot systems.  However, her twelve years as a software engineer (culminating in a IT Architect position at IBM) has made the shortcomings of existing tools apparent.  She designed the circuit board and the software for the eROSI, a novel robot platform [46].  This platform featured an ASCII message-based API available via Bluetooth and images were served wirelessly via UHF channels.  These choices directly affected the platform usability for research students such as the undergraduate team at Berea College [47].  In addition, the PI has either programmed or modified drivers or firmware for robots and devices including the Koala, Finch, Create, Ar.Drone, Lego Mindstorms, Calliope (Dynamixel-based 4 DOF arm), Handyboard, GPS modules, lasers, analog and digital IR and ultrasonic sensors, and custom boards using ATMega chip and serial Bluetooth ICs to interface with Player/Stage, Alice/PREOP, ROS, Carmen, Webots and custom firmware applications.  In addition, the PI has studied open source implementations of other drivers and frameworks including PSOS (Pioneer robots), Tekkotsu, and LCM.  She was the organizer of the 2010 AAAI Robotics Workshop titled ``Enabling Intelligence through Middleware'' an effort to increase awareness of tools within the community and tool needs outside the community [48]. The PI is in a unique position given her expertise, experience and enthusiasm to affect accessibility of robotics controller development within the larger community.

\subsection{Dissemination Plan}
Broad dissemination plans include both the academic and non-academic communities.  Traditional academic dissemination plans include publications and conference presentations in both the educational and robotics software engineering communities.  Future plans include workshops at conference venues as well as at partner campuses will provide hands-on opportunities for those that wish to use the tool for teaching or design.  A website will be maintained and all software will be open source.  However, primary means of dissemination will utilize partnerships with manufacturers or tool developers as we have seen this method is effective at making hardware widely available [42,43].  The PI will also hold hands-on workshops both to inform and to gather feedback from the community.  

\subsection{Methodology}
The research approach consists of iterative design and integrated usability checks.  Rather than attempt to create the research tools in a single iteration, it is better to start with a core of features and iteratively add to those features. This approach allows for the usability testing within the first 18 months as an input into the next iteration.  The schedule details the work for two graduate students for the first three years: one student to focus on grammars and templates and one student to focus on RDSWare and usability testing.  Undergraduate students will assist with coding, testing and support activities.  The project team will meet weekly to discuss progress on features and use cases and any design issues.  Ongoing activities include dissemination through web sites, conferences and site visits.  