
\section*{Data Management Plan}


\subsection*{Types of data produced by this project include:}

{\sc Source Code}

\begin{enumerate}
\item source code for programs that create, validate and use the RDIS data\\
\item grammars and programs that will be used to generate source code for creating, validating and using the RDIS data\\
\item source code for framework integration including ROS nodes, packages and stacks\\
\item  source code for eclipse plug ins\\
\item sample programs, documentation and build files\\
\end{enumerate}



{\sc Curriculum} - curriculum modules for the addition of software to robotics class

{\sc Human Subject Identifiable Data} - completed surveys and interview notes and logs gathered from user studies of RDIS and RDIS-enabled tools 

\subsection*{Standards to be used for data and metadata format and content }

{\sc Source Code:}  The practice of making code available via source code repository is deemed sufficient.  Standard file naming, organization and coding style practices will apply based on the language and platform being targeted.

{\sc Curriculum: } Documents will be made available in PDF, Word or Latex formats.

{\sc Human Subject Identifiable Data:}Survey instruments and interview questions will be made available as part of publications.

\subsection*{Policies for Access and sharing}

{\sc Source Code:} Source code will be available via publicly accessible, web-enabled source control as its results are published.  

{\sc Curriculum:} Curriculum modules will be available on the project website.

{\sc Human Subject Identifiable Data:} Since the data will be from human subjects, appropriate protocols will be instituted to protect individual�s privacy. Evaluation activities will be approved by the University of Alabama Institutional Review Board. Confidentiality of participant information will be placed as a high priority in all evaluation activities. Hence, data accessible to the project and to other researchers will not be identifiable with an individual. All data will be stored using code number identifiers in encrypted files. Access to files will be through passwords provided to appropriate project staff. Data will be stored in locked filed cabinets in a secure site and on portable hard drives by dedicated university encrypted computers at a secure location readily accessible to the project PI and personnel. Raw data will be accessible only to appropriate project staff designated by the PI.

\subsection*{Policies for Re-Use}

{\sc Source code and curriculum materials:} We will release all software artifacts developed in association with this project under the \textit{BSD 3-Clause License} (the ``New BSD'' license) and will make them available via an appropriate outlet such as Google Project Hosting.


{\sc Human subject data} will only be available in aggregated form for the protection of the participants.  Detailed aggregate findings will be available as technical reports and will also be published.

\subsection*{Plans for Archiving Data and for Preservation of Access}
{\sc Personally identifiable data} will be stored in locked file cabinets in a secure site and on portable hard drives by dedicated encrypted computers for 9 years from the beginning of the project and destroyed in a secure manner at the end of that period. Raw data will be accessible only to appropriate project staff designated by the PI. Long term access will be available from secure storage at the University of Alabama upon agreement to a written request sent to the project PI or, if the PI is not available, to the University of Alabama�s Office of Sponsored Programs. Payment to the university of costs incurred for the copying and delivery of data will be required.

{\bf Other Notes:}

We will publish our results in leading conferences and journals. As is typical in the software engineering community, we will rigorously define our experimental procedures. For example, for each case study we will provide the definition and context, including subject software systems, benchmarks, metrics, and the study setting. In addition, we will document our research questions and hypotheses, our data collection and analysis procedures, results, and threats to validity. We will provide access to our data files via online appendices.


We will distribute annual reports via our web site.
