
\section{Broader Impact}\label{sec:bi}

%Broader Impact
The broader impacts of the proposed work can be categorized as either enabling more efficient controller design or enabling new research.   RDIS existence for a given platform will make available a wider range RDIS-enabled frameworks and tools.  Integrated development environments such as Eclipse will be augmented with plug-ins that read the RDIS (from the device or from an online repository) that will present the developer with visual representations of the services and will allow for software creation through component composition for those not interested in using frameworks.  Developers of custom environments can create an RDIS and will have access to standardized tools without needing to write device specific plugins for easy framework.   

The proposed work also enables new research.  The RDIS will change the components in robotics kits so that they can be reconfigured.  This requires additional research in kinematic discovery with low cost components \cite{Croxell2008}.  New instruments for measuring self-efficacy in robotics skills will enable the validation of new tools and techniques in robotics tool development.  Automatic component discovery can be leveraged in other domains where the interfaces are fairly static and there are many data providers and many candidate algorithms.  This research enables a new model of reuse within the research community that could advance validation of published results through sharing of software artifacts. 

Additional broader impacts relate directly to the activities of the PI.  The PI serves as an active mentor within her university and research community.  She advises senior project groups in computer science and electrical engineering.  As a PI in the ARTSI alliance (Advancing Robotics Technology for Societal Impact), she has fostered productive relationships with HBCUs (historically black colleges and universities) that have resulted in talks, workshops, sharing of technical expertise and REU student placement (seven students to date).  Other REUs have been hosted from the Software Engineering REU site at UA and the DREU program. All REUs/undergraduate/graduate advisees receive active mentoring including introductions to graduate faculty at other schools where appropriate, review of application materials, recommendation letters, job reference letters and publication, speaking and attendance opportunities at conferences such as Grace Hopper, Richard Tapia, IJCAI, AAAI, ICRA and IROS.

This project will integrate research and education in a number of ways.
First, the PIs will engage graduate students at UA to support the project.
Through their sustained interaction with the PIs, these new researchers will gain expertise in several aspects of robotics and software engineering, including model-driven engineering.
Thus, one result of this project will be a group of highly trained and qualified researchers who will be capable of sustaining individual research programs in software engineering.
Second, the PIs will integrate the research results from this project into their numerous graduate and upper-level undergraduate courses courses at UA.
When hiring graduate research assistants (GRAs) for this project, we will recruit and encourage persons from underrepresented groups to apply.
PI Anderson and Co-PI Syriani have previous success recruiting and mentoring students from underrepresented groups.  Part of this success is due to the unique status of The University of Alabama among flagship universities as a major recruiter of both minority faculty, administrators, and students (Top 5).



 % %A microprocessor-based software module, RDSConnect, would execute on the computer or SBC to act as both a development and runtime interface for the connected peripherals. In this way, RDSConnect mimics the functionality of firmware to provide a cohesive, discoverable interface to attached components.  RDSConnect interacts with graphical tools that allow building and validation of RDIS files connected using the described communication primitive in real time could be used to integrate a custom microprocessor-based mechanism with supported tools and languages.  This approach is useful for robots composed of communication-enabled parts (i.e. the Calliope with Dynamixel USB-enabled actuators) and allows custom robots built from commodity parts to be easily interfaced to development packages.

%A preliminary specification incorporates differential drive robots and range sensors over usb and serial transports.  A more complete specification will include
%To use any existing robotics software tools, the user must first define the hardware to be programmed.  Misconnects can occur when the definition of data returned or capability does not match the existing hardware.  This step of configuring software to match the hardware is daunting, especially when coupled with the sophistication level required to use existing tool chains.  

%Hardware that can communicate its capabilities and programming API to a general programming tool would offload the knowledge of the hardware to the system of origin (the correct system of record).  In addition, the addition of visual information would support a zero-configuration programming environment that provides fewer frustrations to non-academics or novice programmers.  

%In the addition, the approach of collocating the RDIS on the hardware platform will motivate a new set of tools and bring robotics in line with component design in other areas.  