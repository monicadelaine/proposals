
\section{Broader Impact}\label{sec:bi}

%Broader Impact
The broader impacts of the proposed work can be categorized as either enabling software design by engaging a broader audience or enabling new research.  A new zero-configuration paradigm along with a well-designed, researched graphical programming tool would enable hobbyists to design more sophisticated controllers without requiring expert domain knowledge.  In addition, interactive design environments would enable learning about computer science and computational thinking.  Citizen roboticists are not the only group that benefits.  For example, military personnel are asked to operate and adapt autonomous equipment in the field.  Software that acknowledges the learning curve and written to accommodate the expected sophistication and experiences of the user could increase productivity and reduce frustration.  Students from novice to expert would also be able to leverage tools to learn and create intelligent products that can be personalized to their own needs.  The principles of �Universal Design� [45] in this space may infer that more accessible tools will benefit even advanced users.

The proposed work also enables new research.  The RDIS will change the components in robotics kits so that they can be reconfigured.  This requires additional research in kinematic discovery with low cost components.  New instruments for measuring self-efficacy in robotics skills will enable the validation of new tools and techniques in robotics tool development.  Automatic component discovery can be leveraged in other domains where the interfaces are fairly static and there are many data providers and many candidate algorithms.  This research enables a new model of reuse within the research community that could advance validation of published results through sharing of software artifacts. 

Additional broader impacts relate directly to the activities of the PI. The PI participated in the 2010 NSF CISE Summit on broadening participation, bringing the message and potential outcomes back to her department and college.  The PI serves as an active mentor within her university and research community. She has hosted five REUs from the Software Engineering REU site.  She has also taken many students to conferences including Grace Hopper, Richard Tapia, IJCAI and AAAI.  She also advises a senior project group in electrical engineering tasked with building a chess playing mobile robot from commodity parts (Figure 7).  As a PI in the ARTSI alliance (Advancing Robotics Technology for Societal Impact), she has fostered productive relationships with other partner schools that have resulted in talks, workshops, sharing of technical expertise and REU student placement (seven students to date).  All REUs/undergraduate advisees receive active mentoring including introductions to graduate faculty at other schools where appropriate, review of application materials, recommendation letters, job reference letters and publication and speaking opportunities.  
