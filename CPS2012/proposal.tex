% To generate the summary
%\documentclass[english,10.5pt,letterpaper,twoside]{article}	% specify the class for style
%\usepackage{mathpazo}

\documentclass[english,11pt,letterpaper,twoside]{article}	% specify the class for style
\usepackage[T1]{fontenc}
\usepackage[latin9]{inputenc}
\usepackage{babel}
\usepackage{pslatex}		% use Adobe PostScript font (better than times)

% For URLs in the references
\usepackage{url}

% Advanced Math
\usepackage{amsmath}
\usepackage{amssymb}	% double line font letters (for number sets N,Z,D,Q,R)

% Images and Floats
\usepackage{epsfig}
%\usepackage{subfigure}```
%\usepackage{lscape}			% landscape
%\usepackage{array}			% align vertically in tables
%\usepackage{multirow}

% Theorems, Definitions, ... (always after 'amsmath')
%\usepackage{amsthm}
%\theoremstyle{definition}
%\newtheorem{prop}{Proposition}%[section]
%\newtheorem{lemma}{Lemma}%[section]

% Algorithm environment
%\usepackage{algorithm}
%\usepackage{algorithmic}

% Commutative diagrams
%\usepackage{diagrams}


\usepackage[margin=1in]{geometry}
\setlength\headsep{20pt}

% Reduce the margins and spacings
\parskip 1mm

\frenchspacing	% no double space after end of line period.

% Add a header and footer
\usepackage{fancyhdr}
\pagestyle{fancy}
\fancyhead{}	% clear header
\fancyfoot{}	% clear footer (removes page number!)
% Set the header and footer content
%E: Even page
%O: Odd page
%L: Left field
%C: Center field
%R: Right field
%H: Header
%F: Footer
\usepackage{color}
\definecolor{light-gray}{RGB}{128,128,128}
\fancyhead[CO,CE]{\textit{\textcolor{light-gray}{\small CPS: Breakthrough: Enabling The Discovery And Communication with Robotic Device Interfaces}}}
\fancyfoot[LO,LE]{\textit{\textcolor{light-gray}{\small M. Anderson and E. Syriani}}}
\rfoot{\thepage}
\renewcommand{\headrulewidth}{0.1pt}	% add a line with a specific thickness
\renewcommand{\footrulewidth}{0.1pt}	% add a line with a specific thickness
\newcommand{\monica}[1]{\textbf{\textcolor{red}{#1}}}
\newcommand{\eugene}[1]{\textbf{\textcolor{blue}{#1}}}

%%%%%%%%%%%%%%%%%%%%%%%%%%%%%%%%%%%%%%%%%%%%%

% Standard shortcuts
\newcommand{\eg}{\emph{e.g.,~}}							% exempli gratia (for the sake of example)
\newcommand{\ie}{\emph{i.e.,~}}							% id est (that is)
\newcommand{\etal}{~\emph{et al.}}					% et alia (and others)
\newcommand{\Fig}[1][Figure]{#1~}						% choose Fig. or Figure, depending on the style
\newcommand{\Sect}[1][Section]{#1~}					% section name always with a capital S
\newcommand{\Elem}[1]{\textsf{#1}}					% name of model element
\newcommand{\Lang}[1]{\textit{\textsf{#1}}}	% language, formalism name
\newcommand{\Rule}[1]{\texttt{#1}}					% rule name
\newcommand{\Code}[1]{\textit{\texttt{#1}}}	% inline code
\providecommand{\e}[1]{\ensuremath{\times 10^{#1}}}

%%%%%%%%%%%%%%%%%%%%%%%%%%%%%%%%%%%%%%%%%%%%%

\begin{document}
\fontsize{11.5}{13.5}\selectfont

%\normalem		% disables \emph to underline because overridden in ulem


%\section*{\vspace{-2em}\center Project Summary}

Intro, motivation


\textbf{Intellectual Merit.}
How important is the proposed activity to advancing knowledge and understanding within its own field or across different fields? How well qualified is the proposer (individual or team) to conduct the project? (If appropriate, the reviewer will comment on the quality of the prior work.) To what extent does the proposed activity suggest and explore creative, original, or potentially transformative concepts? How well conceived and organized is the proposed activity? Is there sufficient access to resources?
%
\vspace{-.5\baselineskip}%
\begin{itemize}
	\item \textit{Challenge 1.}

\vspace{-.5\baselineskip}%  
	\item \textit{Challenge 2.}
  
\end{itemize}


\textbf{Broader Impacts.}
How well does the activity advance discovery and understanding while promoting teaching, training, and learning? How well does the proposed activity broaden the participation of underrepresented groups (e.g., gender, ethnicity, disability, geographic, etc.)? To what extent will it enhance the infrastructure for research and education, such as facilities, instrumentation, networks, and partnerships? Will the results be disseminated broadly to enhance scientific and technological understanding? What may be the benefits of the proposed activity to society?

\begin{description}
	\item[Keywords:]
   at least 3 keywords.
\end{description}


\clearpage
\setcounter{page}{1}


\section{Introduction}

The intro

\subsection{Intellectual Merit: ...}

Intellectual merit.


\subsubsection*{Key Research Challenges}

Summary of the challenges.


\subsection{Broader Impacts}

Broader impacts.



\section{Background}\label{sec:rw}

\subsection{Domain modeling of Robot Hardware and Model Metrics}

Not too surprisingly, the Object Management Group (OMG) has taken an interest in standardizing the domain of robotics in the establishment of the Robotics Domain Task Force (RDTF) \cite{omg}. They have a small set of platform independent models (PIM) and platform specific models (PSM) \cite{rtc08,sdo08,rls10} for robotics defined in terms of UML. None, however, focus on the modeling of the kinematics and dynamics of robots. The researchers in \cite{schlegel09,Steck2010} used Eclipse Modeling Framework to deploy application software to different device-framework pairs. One \cite{schlegel09} used models for a P3DX robot and a QFIX-based robot to for several different platforms, including Player, Microsoft Robotics Studio, and SmartSoft \cite{smartsoft}. Statecharts \cite{Harel1988} were used to define robotic behavior. The other \cite{Steck2010} proposed SMARTMARS, a meta-model derived from SMARTSOFT to use for modeling and analysis of robots. A separate researcher \cite{Trojanek2011} used a similar approach to build a subsumption-based robot architecture which was deployed to a NXT Mindstorms robot. These approaches represent an alternative to working directly in a domain-specific language, instead relying on well-understood standards like UML. There are some drawbacks to this approach as at least some textual representation is still needed for the model to be serialized and passed over the network. As well, textual representations tend to simplify visual models as they do not need to include user interfaces aspects of visual models such as the position of a model element on the computer screen. However, structured visual formalisms may be compiled into a domain-specific language which could turn out to be a preferred way to model particular robots.

A secondary goal for this project will involve evaluating the proposed model in reference to other existing models and to assess the progress of the model as it grows in the long-term. Therefore, one shall define a set of metrics to assess model growth and quality, and possibly also the size in relation to other models (other models in this case being the implied models used by the aforementioned frameworks). To achieve a reference frame for how this might be accomplished, a short literature review was conducted on the current view of metrics at the model level, useful for this project. Mohagheghi presents a survey of the state of model metrics as a part of an ongoing research effort in model quality metrics \cite{Mohagheghi2009}. The paper presents six general goals for measurement at the model level which should serve as the inspiration for the metrics used to evaluate a model. The ideas of Lange on model size \cite{Lange2007} are also presented, in particular dealing with UML models. Lange proposes a specific metric for UML relative size, which could be used to measure the size of the proposed robot model against existing description models. In \cite{Wu2010}, the authors present their perception of model metrics for DSLs, offer a classification of the types of effort they believe are encountered when working within a domain-specific model (DSM), the two basic types being Development Effort and Runtime Effort. Within the context of this project, the metrics which measure Development Effort are most interesting, particularly those which measure Modeling Effort, the effort involved in developing a DSML model, and Cognitive Effort, the cognitive effort required to do so. They use a weighted sum to approximate the size of a model as a means to measure modeling effort. This metric should be fairly sufficient for calculating an inexact model size, and it should translate well between different models. Another interesting metric they use is the closeness of mapping ratio (COMR) which they used the ratio of the number of problem-level language primitives to the number of solution-level primitives which do not have mapping to the problem domain. This metric is interesting because it provides insight to the disconnect between the problem domain and the solution domain. However, to actually implement this metric would require a classification of any existing language to be benchmarked against and could be expensive time-wise. The authors also mentioned other metrics which are useful when evaluating imperative languages, however this project will focus on building a declarative language so they will not be so directly applicable. The author in \cite{Sprinkle2010} considers a new approach to estimating a model complexity via the analysis of the meta-model which defines it. This could solve the problem of evaluating complexity in a purely declarative model like RDIS.

\subsection{Hardware descriptions in existing frameworks}
%Many approaches have been used to increase reusability of software artifacts.  Player/Stage drivers can be used directly in ROS \cite{quigley2009}.  Microsoft Robotics Studio uses a service-based paradigm to abstract data sources.  Orocos \cite{Bruyninckx2001} leverages the component design to package related software into a reusable piece.  Carmen \cite{Montemerlo2003}, LCM \cite{Huang2009} and ROS \cite{quigley2009} abstract the message content from the message generation by providing a transport between modules with self-describing data.  These approaches utilize fairly advanced computer science paradigms of distributed development (utilized by many Linux-based open source projects).  Unfamiliarity with these paradigms may discourage many researchers, hobbyists and students from attempting to reuse software, even when it can teach domain specific concepts.

\begin{figure*}[thpb]
      \centering
      \includegraphics[width=5in]{images/URDFPS.pdf}
      \caption{Sample device descriptions within frameworks.}
      \label{psurdf}
\end{figure*}



Many frameworks use a declarative description of the robots.  Player/Stage \cite{vaughan2007} is both a 2D simulator and a robot control framework.  Robot description files are broken into two pieces: 1) a 2D description of the robot and its sensors(Figure \ref{psurdf}: right) and 2) a set of interfaces that abstract the data produced by hardware to a standard format.  The description, used for simulating the robot, consists of a polygon-based footprint with sensor locations marked. Actuation and sensor characteristics along with parameters for simplified error models are used to complete the model of the robot.  A domain-specific set of classes and message types describe what data can be obtained or how the robot can be manipulated including position (pose2d) and distance to other objects (single point is range, multipoint is laser).  The classes and message types represent the interface that abstracts the robot hardware to the data that it can produce or consume.  Writing software to the interfaces that a robot can utilize (rather than the specific robot) allows software to be written either for a simulated robot or a real robot, which in turns eases the transition from simulation to physical implementation.

ROS \cite{quigley2009} targets a 3D simulation framework (Gazebo) and more sophisticated intelligent controller, which require a more rigorous description.  UDRF (Uniform Robot Description Format) provides a 3D physical description broken into links and joints to facilitate not only mobile robots but manipulators as well (Figure \ref{psurdf}: right).  Geometric bounding boxes and meshes allow for collision detection and realistic visualization.  Like Player Stage, ROS utilizes a message-based model to decouple data providers from data producers.  Ideally robots that provide and consume similar data types can be controlled similarly.  Unlike Player Stage, URDF not only serves as a mechanism for simulating robots but also allows for the visualization of real robots in both real-time and off-line (through saved messages). 

URBI \cite{Baillie2005} is a domain specific language for robot applications. It is a platform which controls robots. Its language, urbiscript, has native syntax for parallel and event-driven constructs. This gives it high expressivity for solving problems common in robots. As an example, urbiscript can easily natively express ``when my left arm stops moving, start moving the right arm.'' However, the limitation in URBI is that it is another platform. It expects that the ``UObject'' needed to control the robot exists. Also, URBI focuses only on robotic control, however it can tie into simulators to be used as a control module for the simulation.

The authors in \cite{Schultz2007} present a domain specific language for the control of a novel robot. The ATRON self-configurable robot is composed of several independent modules which can be assembled in different manners to create a robot whose function changes according to how the modules are connected. They use the example of a car configuration in their paper. Their language is role-based, meaning a module can deduce its role by examining its own connectivity. The advantage to this type of description is that it allows the reuse of one behavior description for several configurations of ATRON modules. For example, a description file that represents a car may be used for four-wheel or six-wheel configurations. This is an interesting abstraction for application code, however it has the same problem as URBI, in that a platform-specific module must be created so that a robot may communicate with the framework.  Although the literature reveals very few attempts at using DSLs for hardware device drivers, Thibault et al report the creation of efficient video device drivers using a novel DSL \cite{Thibault1999}.  

A select number of robot control frameworks move beyond visualization information and relevant interface declaration in the hardware description.  PREOP, an Alice-based programming interface \cite{cooper2000,Wellman2009,Anderson} for robots takes this paradigm further.  Not only is 3D visualization information supplied but also the programming interface is completely specified by the selection of the robot object.  This is accomplished by linking the real-time control mechanism and exposed API available to the user within the robot object.  

All robotic architectures require that the user provides information regarding the target hardware devices and the resulting data at development time in absentia of the hardware.  Two problems result from this process: 1) users must understand the hardware and resulting data and 2) encode the hardware details properly in the framework of choice.   A mechanism that provides the user with the details of available resources could remove this step and the associated issues with attempting to properly characterize the hardware.  More importantly, self-describing hardware could lessen both frustration and the need for in-depth hardware knowledge on the part of the user.  

\subsection{RDIS (Robot Device Interface Specification)}
We propose a new paradigm where knowledge of the hardware mechanism is embedded in the hardware, rather than declared in software.  RDIS (Robot Device Interface Specification) has three purposes: 1) provide enough information for simulation and visualization of hardware and controllers, 2) declaratively specify the mechanism for requesting data and actuation, and 3) inform users of standard message types that can be obtained from the hardware to facilitate connection to existing frameworks through discovery.  The challenge in successfully defining the RDIS is in creating a model that captures the generalizable aspects of robots and provides a mechanism to specialize the aspects that vary.

RDIS relies upon the relatively invariant nature of mobile robots.  Although some robots are built for a specific task, general use robots within education and research communities tend to leverage designs that provide closed loop inverse kinematic solutions; differential drive being part of this class of robots.  In addition, many robots including the popular Mobile Robots Pioneer class, iRobot Creates, K-Team robots, Erratic ER-1, White Box Robotics Model 914, Ar.Drones, and BirdBrain Finches
%i don't have them handy..
contain an embedded firmware controller that accepts commands via a serial, Bluetooth, WiFI or USB interface rather than require the users to download a program to onboard memory.  Even robots that require a local software program to run have modes where the local software program presents an API to an external computer (i.e. Lego Mindstorms via Lejos and E-Puck).  In both of these cases, the users create an autonomous controller program that communicates with the firmware to affect actuation and to obtain sensor information.  

%RDIS relies upon two assumptions: 1) robot devices typically contain firmware that manages all hardware resources and exposes access to off-board programs via APIs and 2) many robot devices employ mechanisms that can be generalized and managed at an abstract level.  Not all robot devices contain a firmware that enables off-board programming.  However, this is an assumption that is used within many frameworks (as most frameworks are too processor and memory intensive to run on the limited resources available to on-board controllers).  In addition, firmware often chooses to hide the complexity of managing hardware resources in real time by exposing an API to access and manipulate hardware.

The RDIS specification can be defined as DSL that is used to program robots at a higher, domain-specific level.  In the application of robots, the existence of firmware (not programming hardware directly) simplifies the process.  Designing a declarative specification for robot hardware requires a solid domain model of the hardware connection and services that are provided.  Domain models, when designed properly, can be somewhat invariant to changes and can provide a stable basis for deciding the structure and parameters of the specification.  %Figure \ref{circles} shows a typical implementation of mobile robot device driver.   
%\begin{figure*}[thpb]
%      \centering
%      \includegraphics[width=5in]{scope.pdf}
%      \caption{Scope of initial RDIS implementation.}
%      \label{scope}
%\end{figure*}

The preliminary artifacts generated in this approach include RDIS specifications and grammars that generate a command line program and a ROS driver \cite{Anderson2012}.  The initial scope of this research focuses on a set of domain concepts that represent popular platforms and devices and therefore are present in most frameworks as an abstraction.   The current set of domain concepts include {\sc differential drive} and {\sc range}.   Differential drive robots (robots with two opposing wheels on a single axis) are controllable using linear and rotational velocity.   Acknowledging differential drive explicitly allows for an easier parameterization through state variables that denote odometry resolution, wheel size, offset of axis and wheel base.  While some frameworks explicitly recognize differential drive as a kinematic design (Player), others group all platforms under a single fully expressive interface where the user must know which dimensions are controllable.  For example, within the ROS system, a {\sc Twist} message type is used to provide control commands to a differential driver robot through linear and angular velocity.  The ROS template (discussed in more detail in \cite{Anderson2012}) maps each domain concept to the appropriate programming paradigm, which in this case is a subscription to the {\sc Twist} message and a callback handler that contains the code to map the {\sc Twist} message parameters to the {\sc setSpeed} firmware primitive.   A preliminary RDIS has been implemented for the Finch robot from Bird Brain and the K-Team Koala\cite{Anderson2012}.  In this design, a JSON-based description file is used (although any syntax can be used including XML).  The intermediate product is an abstract syntax tree that represents the robot details in a domain specific model.  This intermediate format can be further processed to verify conformance to the specification.  End products such as framework specific device drivers or stand-alone servers are generated from the verified syntax tree using templates that format data based on the model. 





\section{Research Plan}
\label{sec:research-plan}
%include a plan for validation of the research by experimentation and prototyping;

Implementation Pathways: Embedding in Hardware: Although there are many benefits to utilizing the RDS outside of the hardware’s firmware, the ultimate goal to provide accessible programming is best met through embedding the robot device descriptions within the device.  Discovery occurs when the design environment queries the device for its supported services (or APIs).  The initial approach for platforms that support onboard reconfigurable firmware is to augment the firmware to support a single command that communicates the RDS.  The information provided by the RDS can be used by any RDS-enabled development environment (discussed in the next section).  It is expected that manufacturers will choose to RDS enable their devices once there are more RDS-enabled environments are available.  The Finch designer has already agreed to update the firmware of a couple of select models and has committed to including the command once it is more stable and is being adopted (see attached letter of collaboration).  

Implementation Pathways: RDIS Creation, Validation and Driver Generation Toolset: Even as we work toward adoption by hardware manufacturers, RDS can provide other efficiencies that also serve the purpose of increasing accessibility.   As a side effect of describing the robot device hardware communication interface, the creation of device drivers for frameworks can be automated.   This is not a trivial benefit.  Each runtime framework requires either a custom device driver that links the framework to the device or a bridge between the target framework and a framework that does have device support.   At a minimum, the RDS relieves the manufacturer from providing point solutions for languages, development tools and frameworks.  Target toolsets include C, C++, Java, Python, ROS and Websocket server (provides HTML5 and Javascript support).  The biggest challenge to this task is the cross-platform support of the communication interfaces.  However, consideration for different platforms can be built into the templates using compiler directives.

It is important to note that the RDS toolset is enabled by ANTLR and StringTemplate.  These are open source libraries that parse and process data according to grammars.  These grammars are often used to define domain specific languages that are subsequently processed either by interpreters or translators.  The sample RDS and the translation to a C-based command line controller and a ROS node was achieved through the use of grammars and the ANTLR and StringTemplate libraries.  Although these libraries provide many built-in features, the ability to embed code to customize processing is important to using these tools effectively.

Implementation Pathways: Custom and Reconfigurable Robot Device: In the “Sketching in Hardware” community, many of the robotic devices are custom built by connecting actuators and sensors to netbooks or single board computers (SBCs).  Although the RDS cannot completely specify all possible interactions with every sensor and actuator, the communication to the computer and to the peripheral connection mechanisms can be queried and described.  A microprocessor-based software module, RDSConnect, would execute on the computer or SBC to act as both a development and runtime interface for the connected peripherals. In this way, RDSConnect mimics the functionality of firmware to provide a cohesive, discoverable interface to attached components.  RDSConnect interacts with graphical tools that allow building and validation of RDS files connected using the described communication primitive in real time could be used to integrate a custom microprocessor-based mechanism with supported tools and languages.  This approach is useful for robots composed of communication-enabled parts (i.e. the Calliope with Dynamixel USB-enabled actuators) and allows custom robots built from commodity parts to be easily interfaced to development packages.

Reconfigurable hardware provides a unique challenge.  Rather than a static description, the RDS becomes a mechanism for communicating changes.  As a longer-term goal, research methods that learn the current configuration based on embedded sensors could be appropriately applied here. Either building blocks or configurable linkages would embed proprioceptive sensors to establish kinematic parameters needed for programming.  This kinematic learning can also be accomplished by augmenting existing systems with small standalone, networked sensors that are placed at joints [24]. 

\subsection{Team Member Roles, Responsibilities and Expertise}
%describe the roles, responsibilities, and expertise of the team members, how they cover the set of skills needed to realize the project goals and how their interactions will contribute to integration across core CPS disciplinary areas;

The success of the proposed research requires both breadth and depth in robotics hardware and software.  The PIs main research area centers on distributed robot systems.  However, her twelve years as a software engineer (culminating in a IT Architect position at IBM) has made the shortcomings of existing tools apparent.  She designed the circuit board and the software for the eROSI, a novel robot platform [46].  This platform featured an ASCII message-based API available via Bluetooth and images were served wirelessly via UHF channels.  These choices directly affected the platform’s usability for research students such as the undergraduate team at Berea College [47].  In addition, the PI has either programmed or modified drivers or firmware for robots and devices including the Koala, Finch, Create, Ar.Drone, Lego Mindstorms, Calliope (Dynamixel-based 4 DOF arm), Handyboard, GPS modules, lasers, analog and digital IR and ultrasonic sensors, and custom boards using ATMega chip and serial Bluetooth ICs to interface with Player/Stage, Alice/PREOP, ROS, Carmen, Webots and custom firmware applications.  In addition, the PI has studied open source implementations of other drivers and frameworks including PSOS (Pioneer robots), Tekkotsu, and LCM.  She was the organizer of the 2010 AAAI Robotics Workshop titled “Enabling Intelligence through Middleware”; an effort to increase awareness of tools within the community and tool needs outside the community [48]. The PI is in a unique position given her expertise, experience and enthusiasm to affect accessibility of robotics controller development within the larger community.

\subsection{Dissemination Plan}
Broad dissemination plans include both the academic and non-academic communities.  Traditional academic dissemination plans include publications and conference presentations in both the educational and robotics software engineering communities.  Future plans include workshops at conference venues as well as at partner campuses will provide hands-on opportunities for those that wish to use the tool for teaching or design.  A website will be maintained and all software will be open source.  However, primary means of dissemination will utilize partnerships with manufacturers or tool developers as we have seen this method is effective at making hardware widely available [42,43].  The PI will also hold hands-on workshops both to inform and to gather feedback from the community.  

\subsection{Methodology}
The research approach consists of iterative design and integrated usability checks.  Rather than attempt to create the research tools in a single iteration, it is better to start with a core of features and iteratively add to those features. This approach allows for the usability testing within the first 18 months as an input into the next iteration.  The schedule details the work for two graduate students for the first three years: one student to focus on grammars and templates and one student to focus on RDSWare and usability testing.  Undergraduate students will assist with coding, testing and support activities.  The project team will meet weekly to discuss progress on features and use cases and any design issues.  Ongoing activities include dissemination through web sites, conferences and site visits.  


\section{Curriculum Development Activities}\label{sec:curriculum}

Education, courses



%\section{Schedule}\label{sec:schedule}
\section{Project methodology, management, and dissemination plan}
\subsection{Team Member Roles, Responsibilities and Expertise}
%describe the roles, responsibilities, and expertise of the team members, how they cover the set of skills needed to realize the project goals and how their interactions will contribute to integration across core CPS disciplinary areas;

{\bf need something here about how a domain modeler is important to this research and what tools and expertise is helpful}

The success of the proposed research requires both breadth and depth in robotics hardware and software engineering.  The PIs main research area centers on distributed robot systems.  However, her twelve years as a software engineer (culminating in a IT Architect position at IBM) has made the shortcomings of existing tools apparent.  She designed the circuit board and the software for the eROSI, a novel robot platform \cite{walter2007design}.  This platform featured an ASCII message-based API available via Bluetooth and images were served wirelessly via UHF channels.  These choices directly affected the platform usability for research students such as the undergraduate team at Berea College \cite{Isaacs2006}.  In addition, the PI has either programmed or modified drivers or firmware for robots and devices including the Koala, Finch, Create, Ar.Drone, Lego Mindstorms, Calliope (Dynamixel-based 4 DOF arm), Handyboard, GPS modules, lasers, analog and digital IR and ultrasonic sensors, and custom boards using ATMega chip and serial Bluetooth ICs to interface with Player/Stage, Alice/PREOP, ROS, Carmen, Webots and custom firmware applications.  In addition, the PI has studied open source implementations of other drivers and frameworks including PSOS (Pioneer robots), Tekkotsu, and LCM.  She was the organizer of the 2010 AAAI Robotics Workshop titled ``Enabling Intelligence through Middleware'' an effort to increase awareness of tools within the community and tool needs outside the community \cite{MonicaAnderson2011}. The PI is in a unique position given her expertise, experience and enthusiasm to affect accessibility of robotics controller development within the larger community.

\subsection{Dissemination Plan}
Broad dissemination plans include both the academic and non-academic communities.  Traditional academic dissemination plans include publications and conference presentations in both the educational and robotics software engineering communities.  Future plans include workshops at conference venues as well as at partner campuses will provide hands-on opportunities for those that wish to use the tool for teaching or design.  A website will be maintained and all software will be open source.  However, primary means of dissemination will utilize partnerships with manufacturers or tool developers as we have seen this method is effective at making hardware widely available.  The PI will also hold hands-on workshops both to inform and to gather feedback from the community.  

Additional kinematic designs are motivated by partnerships with NASA and RTP.  NASA is migrating a robotic platform from VxWorks to Ubuntu with a future update to include RTLinux.  This custom kinematic chain involves manipulators with haptic sensors on a MLVDS bus (a new transport).  RTP (Huntsville Research Technology Park) is an initiative to by the governor of Alabama to create a National Robotics Technology Development and Training Center at Calhoun Community College.  Many robot platforms have been purchased and donated for training and are available for evaluation as candidate platforms for RDIS inclusion.  Both opportunities expands the platforms that can modeled and contribute to the domain model that can be generalized across other platforms. 

\subsection{Methodology}
The research approach consists of iterative design and integrated usability checks.  Rather than attempt to create the research tools in a single iteration, it is better to start with a core of features and iteratively add to those features. This approach allows for the usability testing within the first 18 months as an input into the next iteration.  The schedule details the work for two graduate students for the three years: one student to focus on analysis of platforms and the evolution of RDIS and one student to focus on the graphical tool for building and incorporating the RDIS into tools and frameworks.  Undergraduate students will assist with coding, testing and support activities.  The project team will meet weekly to discuss progress on features and use cases and any design issues.  Ongoing activities include dissemination through web sites, conferences and site visits.    

\subsection{Tasks and timeline}
% Requires the booktabs if the memoir class is not being used
\begin{table}[htbp]
   \centering
   \begin{tabular}{|l|l|} 
      Quarter    & Task \\
      Q4 2012       &   \\
      Q1 2013       & \\
      Q2 2013       & \\
      Q3 2013       & \\
      Q4 2013       & \\
      Q1 2014       & \\
      Q2 2014       & \\
      Q3 2014       & \\
      Q4 2014       & \\
      Q1 2015       & \\
      Q2 2015       & \\
      Q3 2015       & \\
   \end{tabular}
   \caption{Timeline of proposed tasks.}
   \label{tab:tasks}
\end{table}



\section{Intellectual Merit}\label{sec:im}

%Intellectual Merit
The intellectual merit of the proposed work centers on research that enables non-academics and non-roboticists to interactively design intelligent robotics controllers.  Three areas need to be addressed: abstract the hardware from the software by making the hardware the ``system of record'', provide tools that encourage interactive design experiences and identify and validate correct tool design principles while investigating appropriate GUI-based design strategies that manage complexity.  

To use any existing robotics software tools, the user must first define the hardware to be programmed.  Misconnects can occur when the definition of data returned or capability does not match the existing hardware.  This step of configuring software to match the hardware is daunting, especially when coupled with the sophistication level required to use existing tool chains.  Hardware that can communicate its capabilities and programming API to a general programming tool would offload the knowledge of the hardware to the system of origin (the correct system of record).  In addition, the addition of visual information would support a zero-configuration programming environment that provides fewer frustrations to non-academics or novice programmers.  This aspect of the proposed project is transformative; it could potentially affect the creation of new hardware platforms, reduce the complexity of device drivers (encourage new devices) and lower the learning curve to working with platforms.  Although initial iterations will focus on static hardware designs, the application to reconfigurable robots could enable ``on-the-fly'' physical reconfiguration with reasonable ability to program new capabilities in the field.

Reconfigurable hardware provides a unique challenge.  Rather than a static description, the RDIS becomes a mechanism for communicating changes.  As a longer-term goal, research methods that learn the current configuration based on embedded sensors could be appropriately applied here. Either building blocks or configurable linkages would embed proprioceptive sensors to establish kinematic parameters needed for programming.  This kinematic learning can also be accomplished by augmenting existing systems with small standalone, networked sensors that are placed at joints [24]. 

Although there are many benefits to utilizing the RDIS outside of the hardware� the ultimate goal  is best met through embedding the robot device descriptions within the device.  Discovery occurs when the design environment queries the device for its supported services (or APIs).  The initial approach for platforms that support onboard reconfigurable firmware is to augment the firmware to support a single command that communicates the RDIS.  The information provided by the RDIS can be used by any RDIS-enabled development environment.  It is expected that manufacturers will choose to RDIS enable their devices once there are more RDIS-enabled environments are available.  The Finch designer has already agreed to update the firmware of a couple of select models and has committed to including the command once it is more stable and is being adopted.  In the ``Sketching in Hardware'' community, many of the robotic devices are custom built by connecting actuators and sensors to netbooks or single board computers (SBCs).  Although the RDIS cannot completely specify all possible interactions with every sensor and actuator, the communication to the computer and to the peripheral connection mechanisms can be queried and described.  %A microprocessor-based software module, RDSConnect, would execute on the computer or SBC to act as both a development and runtime interface for the connected peripherals. In this way, RDSConnect mimics the functionality of firmware to provide a cohesive, discoverable interface to attached components.  RDSConnect interacts with graphical tools that allow building and validation of RDIS files connected using the described communication primitive in real time could be used to integrate a custom microprocessor-based mechanism with supported tools and languages.  This approach is useful for robots composed of communication-enabled parts (i.e. the Calliope with Dynamixel USB-enabled actuators) and allows custom robots built from commodity parts to be easily interfaced to development packages.




\section{Broader Impact}\label{sec:bi}

%Broader Impact
The broader impacts of the proposed work can be categorized as either enabling more efficient controller design or enabling new research.   RDIS existence for a given platform will make available a wider range RDIS-enabled frameworks and tools.  Integrated development environments such as Eclipse will be augmented with plug-ins that read the RDIS (from the device or from an online repository) that will present the developer with visual representations of the services and will allow for software creation through component composition for those not interested in using frameworks.  Developers of custom environments can create an RDIS and will have access to standardized tools without needing to write device specific plugins for easy framework.  Developers can use the error information propagated from RDIS as a starting point for modeling error co-variances.  

The proposed work also enables new research.  The RDIS will change the components in robotics kits so that they can be reconfigured.  This requires additional research in kinematic discovery with low cost components \cite{Croxell2008}.  New instruments for measuring self-efficacy in robotics skills will enable the validation of new tools and techniques in robotics tool development.  Automatic component discovery can be leveraged in other domains where the interfaces are fairly static and there are many data providers and many candidate algorithms.  This research enables a new model of reuse within the research community that could advance validation of published results through sharing of software artifacts. 

Additional broader impacts relate directly to the activities of the PI.  The PI serves as an active mentor within her university and research community.  She advises senior project groups in computer science and electrical engineering.  As a PI in the ARTSI alliance (Advancing Robotics Technology for Societal Impact), she has fostered productive relationships with HBCUs (historically black colleges and universities) that have resulted in talks, workshops, sharing of technical expertise and REU student placement (seven students to date).  Other REUs have been hosted from the Software Engineering REU site at UA and the DREU program. All REUs/undergraduate/graduate advisees receive active mentoring including introductions to graduate faculty at other schools where appropriate, review of application materials, recommendation letters, job reference letters and publication, speaking and attendance opportunities at conferences such as Grace Hopper, Richard Tapia, IJCAI, AAAI, ICRA and IROS.

This project will integrate research and education in a number of ways.
First, the PIs will engage graduate students at UA to support the project.
Through their sustained interaction with the PIs, these new researchers will gain expertise in several aspects of robotics and software engineering, including model-driven engineering.
Thus, one result of this project will be a group of highly trained and qualified researchers who will be capable of sustaining individual research programs in software engineering.
Second, the PIs will integrate the research results from this project into their numerous graduate and upper-level undergraduate courses courses at UA.
When hiring graduate research assistants (GRAs) for this project, we will recruit and encourage persons from underrepresented groups to apply.
PI Anderson and Co-PI Syriani have previous success recruiting and mentoring students from underrepresented groups.
\monica{african american and female students???}


 % %A microprocessor-based software module, RDSConnect, would execute on the computer or SBC to act as both a development and runtime interface for the connected peripherals. In this way, RDSConnect mimics the functionality of firmware to provide a cohesive, discoverable interface to attached components.  RDSConnect interacts with graphical tools that allow building and validation of RDIS files connected using the described communication primitive in real time could be used to integrate a custom microprocessor-based mechanism with supported tools and languages.  This approach is useful for robots composed of communication-enabled parts (i.e. the Calliope with Dynamixel USB-enabled actuators) and allows custom robots built from commodity parts to be easily interfaced to development packages.

%A preliminary specification incorporates differential drive robots and range sensors over usb and serial transports.  A more complete specification will include
%To use any existing robotics software tools, the user must first define the hardware to be programmed.  Misconnects can occur when the definition of data returned or capability does not match the existing hardware.  This step of configuring software to match the hardware is daunting, especially when coupled with the sophistication level required to use existing tool chains.  

%Hardware that can communicate its capabilities and programming API to a general programming tool would offload the knowledge of the hardware to the system of origin (the correct system of record).  In addition, the addition of visual information would support a zero-configuration programming environment that provides fewer frustrations to non-academics or novice programmers.  

%In the addition, the approach of collocating the RDIS on the hardware platform will motivate a new set of tools and bring robotics in line with component design in other areas.  


\section{Results from Prior NSF Support}\label{sec:prior-nsf}
\begin{enumerate}
\item Monica Anderson - PI for the NSF Grant: CCLI: Enhancing student motivation in the first year computing science curriculum (DUE- 0736789), 04/01/2008-08/31/2011.  This is an ongoing project that addresses retention issues due to motivation through specially designed modules that use PREOP in a CS0 or CS1 laboratory setting.  We provided PREOP workshops to teachers at the ARTSI Faculty Meeting, demonstrated PREOP at the Google education summit and added PREOP to Alice instructor training (July 2011). Three papers have been published including two conference papers and two journal papers \cite{Davis2009,wellman2009alice,Wellman2009b,Anderson}.  

\item Monica Anderson - PI for the Grant: ARTSI-Advancing Robotic Technology for Societal Impact (BPC-A- 0742123), 09/01/07-12/31/2012. This is an ongoing broadening participation grant that increases the number of students interested in computer science and robotics through both research and creative experiences. Robotics programs have been established or enhanced at 14 HBCUs. The PI has provided each school with both onsite and remote instructional support ranging from hands-on workshops on open-source robotics development environments and controller development to hardware assembly instructions. Seven summer research assistants have been hosted at UA, with one student contributing to an accepted paper \cite{Wellman2009} and resulting in numerous poster presentations.  
\item Monica Anderson - PI for the Grant: SGER: Impact of Intermittent Networks on Team-based Coverage (IIS-0846976), 09/01/2008-08/31/2009. This project investigated both surveillance in constrained environments using limited resources and methods for coordinating robot teams with limited communications capability. The novel approach of using ALUL and a QOS metric allowed for balanced target acquisition and tracking \cite{Veluchamy2010,Alexander2009,Mckenzie2010,McKenzie2009}.  

\item Monica Anderson - Co-PI for the Grant: EMT: Collaborative Research: Primate-inspired Heterogeneous Mobile and Static Sensor Networks (CNS: 0829827) 09/01/2008-08/31/2011. This project investigates bio-inspired approaches for multirobot coordination. We show that given a greedy approach to selecting the next point to search, observation information provides more relevant input to the next action selected. Experimental results show that using observation to infer state and intent to disperse nodes performs better than non-communicative and potential field based cooperation and in some cases direct communications. \cite{Wellman2009d,Dawson2011,Dawson2010,Dawson2011b,Wellman2009c}.  

\item Monica Anderson - PI for the Grant: University of Alabama Participation of EPSCOR Institution in SSR-RC: QOS service metrics for unmanned surveillance (IIP: 1026528).  06/15/2010-06/14/2012. We consider ALUL as a metric that indicates the target acquisition ability of a particular system configuration. When used in conjunction with temporal constraints, we can intelligently automate camera coverage to improve target acquisition and tracking performance of the system. The results from the experiments confirm that the coverage in constrained environments when the existing camera configuration cannot view large portions of the region of our interest improves when ALUL is considered.  To date, one conference paper and one journal article have been accepted \cite{Veluchamy2011,Dukeman2011}.
\item Eugene Syriani - New investigator and has no prior NSF support.

\end{enumerate}


\clearpage
\setcounter{page}{1}

%%%%%%%%%%%%%%%%%%%%%%%%%%%%%%%%%%%%%%%%%%%%%%%%%%%%%%%%%%%%%%%%%%%

% Specify the bibliography style and data
\bibliographystyle{IEEEtran}
\bibliography{anderson,syriani}

\clearpage


\section*{Data Management Plan}


\subsection*{Types of data produced by this project include:}

Source code artifacts include: 1)source code for programs that create, validate and use the RDIS data, 2) grammars and programs that will be used to generate source code for creating, validating and using the RDIS data, 3) source code for framework integration including ROS nodes, packages and stacks, 4) source code for eclipse plug ins, and 5) sample programs, documentation and build files.  Curriculum modules include those created for the domain modeling and robotics courses.  Human Subject Identifiable Data include completed surveys and interview notes and logs gathered from user studies of RDIS and RDIS-enabled tools 

\subsection*{Standards to be used for data and metadata format and content }

\subsubsection*{Source Code:}  The practice of making code available via source code repository is deemed sufficient.  Standard file naming, organization and coding style practices will apply based on the language and platform being targeted.

\subsubsection*{Curriculum: } Documents will be made available in PDF, Word or Latex formats.

\subsubsection*{ Human Subject Identifiable Data:}Survey instruments and interview questions will be made available as part of publications.

\subsection*{Policies for Access and sharing}

\subsubsection*{Source Code:} Source code will be available via publicly accessible, web-enabled source control as its results are published.  

\subsubsection*{Curriculum:} Curriculum modules will be available on the project website.

\subsubsection*{Human Subject Identifiable Data:} Since the data will be from human subjects, appropriate protocols will be instituted to protect individual's privacy. Evaluation activities will be approved by the University of Alabama Institutional Review Board. Confidentiality of participant information will be placed as a high priority in all evaluation activities. Hence, data accessible to the project and to other researchers will not be identifiable with an individual. All data will be stored using code number identifiers in encrypted files. Access to files will be through passwords provided to appropriate project staff. Data will be stored in locked filed cabinets in a secure site and on portable hard drives by dedicated university encrypted computers at a secure location readily accessible to the project PI and personnel. Raw data will be accessible only to appropriate project staff designated by the PI.

\subsection*{Policies for Re-Use}

\subsubsection*{Source code and curriculum materials:} We will release all software artifacts developed in association with this project under the \textit{BSD 3-Clause License} (the ``New BSD'' license) and will make them available via an appropriate outlet such as Google Project Hosting.


\subsubsection*{Human subject data} Human subject data will only be available in aggregated form for the protection of the participants.  Detailed aggregate findings will be available as technical reports and will also be published.

\subsection*{Plans for Archiving Data and for Preservation of Access}
\subsubsection*{Personally identifiable data} Personally identifiable data will be stored in locked file cabinets in a secure site and on portable hard drives by dedicated encrypted computers for 9 years from the beginning of the project and destroyed in a secure manner at the end of that period. Raw data will be accessible only to appropriate project staff designated by the PI. Long term access will be available from secure storage at the University of Alabama upon agreement to a written request sent to the project PI or, if the PI is not available, to the University of Alabama's Office of Sponsored Programs. Payment to the university of costs incurred for the copying and delivery of data will be required.

\subsection*{Other Notes:}

We will publish our results in leading conferences and journals as well as make results and methodologies available via the CPS Virtual Organization (PI is already a member). As is typical in the software engineering community, we will rigorously define our experimental procedures. For example, for each case study we will provide the definition and context, including subject software systems, benchmarks, metrics, and the study setting. In addition, we will document our research questions and hypotheses, our data collection and analysis procedures, results, and threats to validity. We will provide access to our data files via online appendices.  We will distribute annual reports via our web site.


\end{document}
