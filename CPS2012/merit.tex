
\section{Intellectual Merit}\label{sec:im}

%Intellectual Merit
RDIS provides a mechanism for centralizing the knowledge of the syntax (how to access the service) and semantics (what is the meaning of the input/output of the service) to the embedded system itself through a fully descriptive, declarative language.   To accomplish this, we propose an iterative approach that builds a domain model that will be modified to support an increasing larger set of popular sensors and kinematic designs.   This domain model will drive the evolution of the RDIS and accompanying toolsets to greatly improve the ability of developers and researchers to incorporate new platforms and sensors in a wider range of frameworks.  This aspect of the proposed project is transformative; it could potentially affect the creation of new hardware platforms, reduce the complexity of device drivers (encourage new devices) and lower the learning curve to working with platforms. 

Although there are many benefits to utilizing the RDIS outside of the hardware the ultimate goal  is best met by embedding the robot device descriptions within the device.  Discovery occurs when the design environment queries the device for its supported services or API.  The initial approach for platforms that support onboard reconfigurable firmware is to augment the firmware to support a single command that communicates the RDIS.  The information provided by the RDIS can be used by any RDIS-enabled development environment.  It is expected that manufacturers will choose RDIS to enable their devices once there are more RDIS-enabled environments available.  The Finch designer has already agreed to update the firmware of a couple of select models and has committed to including the command once it is more stable and is being adopted.  In the ``Sketching in Hardware'' community, many of the robotic devices are custom built by connecting actuators and sensors to netbooks or single board computers (SBCs).  Although RDIS cannot completely specify all possible interactions with every sensor and actuator, the communication to the computer and to the peripheral connection mechanisms can be queried and described. 

Reconfigurable hardware provides a unique challenge.  Rather than a static description, RDIS becomes a mechanism for communicating changes.  As a longer-term goal, research methods that learn the current configuration based on embedded sensors could be appropriately applied here. Either building blocks or configurable linkages would embed proprioceptive sensors to establish kinematic parameters needed for programming.  This kinematic learning can also be accomplished by augmenting existing systems with small standalone, networked sensors that are placed at joints.   Although initial iterations will focus on static hardware designs, the application to reconfigurable robots could enable ``on-the-fly'' physical reconfiguration with reasonable ability to program new capabilities in the field.
