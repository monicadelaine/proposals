
\section{Intellectual Merit}\label{sec:im}

%Intellectual Merit
The intellectual merit of the proposed work centers on research that enables non-academics and non-roboticists to interactively design intelligent robotics controllers.  Three areas need to be addressed: abstract the hardware from the software by making the hardware the ``system of record'', provide tools that encourage interactive design experiences and identify and validate correct tool design principles while investigating appropriate GUI-based design strategies that manage complexity.  

To use any existing robotics software tools, the user must first define the hardware to be programmed.  Misconnects can occur when the definition of data returned or capability does not match the existing hardware.  This step of configuring software to match the hardware is daunting, especially when coupled with the sophistication level required to use existing tool chains.  Hardware that can communicate its capabilities and programming API to a general programming tool would offload the knowledge of the hardware to the system of origin (the correct system of record).  In addition, the addition of visual information would support a zero-configuration programming environment that provides fewer frustrations to non-academics or novice programmers.  This aspect of the proposed project is transformative; it could potentially affect the creation of new hardware platforms, reduce the complexity of device drivers (encourage new devices) and lower the learning curve to working with platforms.  Although initial iterations will focus on static hardware designs, the application to reconfigurable robots could enable ``on-the-fly'' physical reconfiguration with reasonable ability to program new capabilities in the field.

Reconfigurable hardware provides a unique challenge.  Rather than a static description, the RDIS becomes a mechanism for communicating changes.  As a longer-term goal, research methods that learn the current configuration based on embedded sensors could be appropriately applied here. Either building blocks or configurable linkages would embed proprioceptive sensors to establish kinematic parameters needed for programming.  This kinematic learning can also be accomplished by augmenting existing systems with small standalone, networked sensors that are placed at joints [24]. 

Although there are many benefits to utilizing the RDIS outside of the hardware� the ultimate goal  is best met through embedding the robot device descriptions within the device.  Discovery occurs when the design environment queries the device for its supported services (or APIs).  The initial approach for platforms that support onboard reconfigurable firmware is to augment the firmware to support a single command that communicates the RDIS.  The information provided by the RDIS can be used by any RDIS-enabled development environment.  It is expected that manufacturers will choose to RDIS enable their devices once there are more RDIS-enabled environments are available.  The Finch designer has already agreed to update the firmware of a couple of select models and has committed to including the command once it is more stable and is being adopted.  In the ``Sketching in Hardware'' community, many of the robotic devices are custom built by connecting actuators and sensors to netbooks or single board computers (SBCs).  Although the RDIS cannot completely specify all possible interactions with every sensor and actuator, the communication to the computer and to the peripheral connection mechanisms can be queried and described.  %A microprocessor-based software module, RDSConnect, would execute on the computer or SBC to act as both a development and runtime interface for the connected peripherals. In this way, RDSConnect mimics the functionality of firmware to provide a cohesive, discoverable interface to attached components.  RDSConnect interacts with graphical tools that allow building and validation of RDIS files connected using the described communication primitive in real time could be used to integrate a custom microprocessor-based mechanism with supported tools and languages.  This approach is useful for robots composed of communication-enabled parts (i.e. the Calliope with Dynamixel USB-enabled actuators) and allows custom robots built from commodity parts to be easily interfaced to development packages.

